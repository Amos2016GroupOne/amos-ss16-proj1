\documentclass[]{article}
\usepackage{lmodern}
\usepackage{amssymb,amsmath}
\usepackage{tikz-uml}
\usepackage{ifxetex,ifluatex}
\usepackage{fixltx2e} % provides \textsubscript
\ifnum 0\ifxetex 1\fi\ifluatex 1\fi=0 % if pdftex
  \usepackage[T1]{fontenc}
  \usepackage[utf8]{inputenc}
\else % if luatex or xelatex
  \ifxetex
    \usepackage{mathspec}
    \usepackage{xltxtra,xunicode}
  \else
    \usepackage{fontspec}
  \fi
  \defaultfontfeatures{Mapping=tex-text,Scale=MatchLowercase}
  \newcommand{\euro}{€}
\fi
% use upquote if available, for straight quotes in verbatim environments
\IfFileExists{upquote.sty}{\usepackage{upquote}}{}
% use microtype if available
\IfFileExists{microtype.sty}{\usepackage{microtype}}{}
\usepackage{longtable,booktabs}
\usepackage{graphicx}
\makeatletter
\def\maxwidth{\ifdim\Gin@nat@width>\linewidth\linewidth\else\Gin@nat@width\fi}
\def\maxheight{\ifdim\Gin@nat@height>\textheight\textheight\else\Gin@nat@height\fi}
\makeatother
% Scale images if necessary, so that they will not overflow the page
% margins by default, and it is still possible to overwrite the defaults
% using explicit options in \includegraphics[width, height, ...]{}
\setkeys{Gin}{width=\maxwidth,height=\maxheight,keepaspectratio}
\ifxetex
  \usepackage[setpagesize=false, % page size defined by xetex
              unicode=false, % unicode breaks when used with xetex
              xetex]{hyperref}
\else
  \usepackage[unicode=true]{hyperref}
\fi
\hypersetup{breaklinks=true,
            bookmarks=true,
            pdfauthor={},
            pdftitle={arc42 Template},
            colorlinks=true,
            citecolor=blue,
            urlcolor=blue,
            linkcolor=magenta,
            pdfborder={0 0 0}}
\urlstyle{same}  % don't use monospace font for urls
\setlength{\parindent}{0pt}
\setlength{\parskip}{6pt plus 2pt minus 1pt}
\setlength{\emergencystretch}{3em}  % prevent overfull lines
\setcounter{secnumdepth}{0}

\title{\includegraphics{images/arc42-logo.png} Template}
\date{2016-06-16}

\begin{document}
\maketitle

\section{About arc42}

arc42, the Template for documentation of software and system
architecture.

By Dr. Gernot Starke, Dr. Peter Hruschka and contributors.

Template Revision: 6.5 EN (based on asciidoc), Juni 2014

© We acknowledge that this document uses material from the arc 42
architecture template, \url{http://www.arc42.de}. Created by Dr. Peter
Hruschka \& Dr. Gernot Starke. For additional contributors see
\url{http://arc42.de/sonstiges/contributors.html}

\begin{quote}
\textbf{Note}

This version of the template contains some help and explanations. It is
used for familiarization with arc42 and the understanding of the
concepts. For documentation of your own system you use better the
\emph{plain} version.
\end{quote}

\section{Introduction and Goals}

The introduction to the architecture documentation should list the
driving forces that software architects must consider in their
decisions. This includes on the one hand the fulfillment of functional
requirements of the stakeholders, on the other hand the fulfillment of
or compliance with required constraints, always in consideration of the
architecture goals.

\subsection{Requirements Overview}

\subsection{Quality Goals}

\subsection{Stakeholders}

\section{Architecture Constraints}

\subsection{Technical Constraints}

\begin{longtable}[c]{@{}ll@{}}
\toprule\addlinespace
\begin{minipage}[b]{0.19\columnwidth}\raggedright
Technical Constraints
\end{minipage}
\\\addlinespace
\midrule\endhead
\begin{minipage}[t]{0.19\columnwidth}\raggedright
\emph{Hardware Constraints}
\end{minipage}
\\\addlinespace
\begin{minipage}[t]{0.19\columnwidth}\raggedright
C1
\end{minipage} & \begin{minipage}[t]{0.75\columnwidth}\raggedright
insert description here
\end{minipage}
\\\addlinespace
\begin{minipage}[t]{0.19\columnwidth}\raggedright
C2
\end{minipage} & \begin{minipage}[t]{0.75\columnwidth}\raggedright
insert description here
\end{minipage}
\\\addlinespace
\begin{minipage}[t]{0.19\columnwidth}\raggedright
C3
\end{minipage} & \begin{minipage}[t]{0.75\columnwidth}\raggedright
insert description here
\end{minipage}
\\\addlinespace
\begin{minipage}[t]{0.19\columnwidth}\raggedright
\emph{Software Constraints}
\end{minipage}
\\\addlinespace
\begin{minipage}[t]{0.19\columnwidth}\raggedright
C4
\end{minipage} & \begin{minipage}[t]{0.75\columnwidth}\raggedright
insert description here
\end{minipage}
\\\addlinespace
\begin{minipage}[t]{0.19\columnwidth}\raggedright
C5
\end{minipage} & \begin{minipage}[t]{0.75\columnwidth}\raggedright
insert description here
\end{minipage}
\\\addlinespace
\begin{minipage}[t]{0.19\columnwidth}\raggedright
C6
\end{minipage} & \begin{minipage}[t]{0.75\columnwidth}\raggedright
insert description here
\end{minipage}
\\\addlinespace
\begin{minipage}[t]{0.19\columnwidth}\raggedright
\emph{Operating System Constraints}
\end{minipage}
\\\addlinespace
\begin{minipage}[t]{0.19\columnwidth}\raggedright
C7
\end{minipage} & \begin{minipage}[t]{0.75\columnwidth}\raggedright
insert description here
\end{minipage}
\\\addlinespace
\begin{minipage}[t]{0.19\columnwidth}\raggedright
C8
\end{minipage} & \begin{minipage}[t]{0.75\columnwidth}\raggedright
insert description here
\end{minipage}
\\\addlinespace
\begin{minipage}[t]{0.19\columnwidth}\raggedright
C9
\end{minipage} & \begin{minipage}[t]{0.75\columnwidth}\raggedright
insert description here
\end{minipage}
\\\addlinespace
\begin{minipage}[t]{0.19\columnwidth}\raggedright
\emph{Programming Constraints}
\end{minipage}
\\\addlinespace
\begin{minipage}[t]{0.19\columnwidth}\raggedright
C10
\end{minipage} & \begin{minipage}[t]{0.75\columnwidth}\raggedright
insert description here
\end{minipage}
\\\addlinespace
\begin{minipage}[t]{0.19\columnwidth}\raggedright
C11
\end{minipage} & \begin{minipage}[t]{0.75\columnwidth}\raggedright
insert description here
\end{minipage}
\\\addlinespace
\begin{minipage}[t]{0.19\columnwidth}\raggedright
C12
\end{minipage} & \begin{minipage}[t]{0.75\columnwidth}\raggedright
insert description here
\end{minipage}
\\\addlinespace
\bottomrule
\addlinespace
\caption{List of Technical Constraints}
\end{longtable}

\subsection{Organizational Constraints}

\begin{longtable}[c]{@{}ll@{}}
\toprule\addlinespace
\begin{minipage}[b]{0.19\columnwidth}\raggedright
Organizational Constraints
\end{minipage}
\\\addlinespace
\midrule\endhead
\begin{minipage}[t]{0.19\columnwidth}\raggedright
\emph{Organization and Structure}
\end{minipage}
\\\addlinespace
\begin{minipage}[t]{0.19\columnwidth}\raggedright
C1
\end{minipage} & \begin{minipage}[t]{0.75\columnwidth}\raggedright
insert description here
\end{minipage}
\\\addlinespace
\begin{minipage}[t]{0.19\columnwidth}\raggedright
C2
\end{minipage} & \begin{minipage}[t]{0.75\columnwidth}\raggedright
insert description here
\end{minipage}
\\\addlinespace
\begin{minipage}[t]{0.19\columnwidth}\raggedright
\emph{Resources (Budget, Time, Personnel)}
\end{minipage}
\\\addlinespace
\begin{minipage}[t]{0.19\columnwidth}\raggedright
C3
\end{minipage} & \begin{minipage}[t]{0.75\columnwidth}\raggedright
insert description here
\end{minipage}
\\\addlinespace
\begin{minipage}[t]{0.19\columnwidth}\raggedright
C4
\end{minipage} & \begin{minipage}[t]{0.75\columnwidth}\raggedright
insert description here
\end{minipage}
\\\addlinespace
\begin{minipage}[t]{0.19\columnwidth}\raggedright
\emph{Organizational Standards}
\end{minipage}
\\\addlinespace
\begin{minipage}[t]{0.19\columnwidth}\raggedright
C5
\end{minipage} & \begin{minipage}[t]{0.75\columnwidth}\raggedright
insert description here
\end{minipage}
\\\addlinespace
\begin{minipage}[t]{0.19\columnwidth}\raggedright
C6
\end{minipage} & \begin{minipage}[t]{0.75\columnwidth}\raggedright
insert description here
\end{minipage}
\\\addlinespace
\begin{minipage}[t]{0.19\columnwidth}\raggedright
\emph{Legal Factors}
\end{minipage}
\\\addlinespace
\begin{minipage}[t]{0.19\columnwidth}\raggedright
C7
\end{minipage} & \begin{minipage}[t]{0.75\columnwidth}\raggedright
insert description here
\end{minipage}
\\\addlinespace
\begin{minipage}[t]{0.19\columnwidth}\raggedright
C8
\end{minipage} & \begin{minipage}[t]{0.75\columnwidth}\raggedright
insert description here
\end{minipage}
\\\addlinespace
\bottomrule
\addlinespace
\caption{List of Organizational Constraints}
\end{longtable}

\subsection{Conventions}

\section{System Scope and Context}

\subsection{Business Context}

\subsection{Technical Context}

\subsection{External Interfaces}

\section{Solution Strategy}

\section{Building Block View}

\subsection{Level 1}

The following diagram shows the main building blocks of the system and
their interdependencies:
\begin{figure}
\center
\begin{tikzpicture}
\begin{umlcomponent}{BLE Remote}

\umlbasiccomponent{TagCtrl}
\umlbasiccomponent[x=4]{GraphCtrl}
\umlbasiccomponent[x=8]{SettingsCtrl}

\umlbasiccomponent[x=4,y=-3]{SettingsService}
\umlbasiccomponent[x=0,y=-3]{DataStorage}
\umldep{TagCtrl}{DataStorage}
\umldep{GraphCtrl}{DataStorage}

\umldep{TagCtrl}{SettingsService}
\umldep{SettingsCtrl}{SettingsService}

\end{umlcomponent}
\end{tikzpicture}
\caption{System Building Block View}
\end{figure}
Comments regarding structure and interdependencies at Level 1:

\subsubsection{TagCtrl (Black Box Description)}

The TagCtrl subsystem handles all interactions with the Sensor Tags. It exposes some of those methods to the user interface so the user can trigger scanning and connection with the tags. 

It retrieves specific configuration like scan duration from the SettingsService and stores data it gets from the tags in the DataStorage 

\subsubsection{GraphCtrl (Black Box Description)}

The GraphCtrl subsystem handles all interaction with the information graph. It exposes certain functions to the user interface so the user can interact with the graph. 

It retrieves the data to display from the DataStorage.


\subsubsection{SettingsCtrl (Black Box Description)}

The SettingsCtrl subsystem handles the display of current preferences and allows the user to change these settings. It stores and retrieves the settings from the SettingsService.


\subsubsection{DataStorage (Black Box Description)}

The DataStorage subsystem is designed to store data in a structured way and allows for easy storage and retrieval of said data. 

\subsubsection{SettingsService (Black Box Description)}

The SettingsService subsystem allows for storage and retrieval of the settings. It is accessed by multiple parts of the application to provide a way for users to set custom preferences.

\subsubsection{Open Issues}

\subsection{Level 2}

\subsubsection{Building Block Name 1 (White Box Description)}

\textless{}insert diagram of building block 1 here\textgreater{}

\paragraph{Building Block Name 1.1 (Black Box Description)}

\paragraph{Building Block Name 1.2 (Black Box Description)}

Structure according to black box template

\paragraph{\ldots{}}

\paragraph{Building Block Name 1.n (Black Box Description)}

\paragraph{Description of Relationships}

\paragraph{Open Issues}

\subsubsection{Building Block Name 2 (White Box Description)}

\ldots{}

\textless{}insert diagram of building block 2 here\textgreater{}

\paragraph{Building Block Name 2.1 (Black Box Description)}

Structure according to black box template

\paragraph{Building Block Name 2.2 (Black Box Description)}

\paragraph{\ldots{}}

\paragraph{Building Block Name 2.n (Black Box Description)}

\paragraph{Description of Relationships}

\paragraph{Open Issues}

\subsubsection{Building Block Name 3 (White Box Description)}

\ldots{}

\textless{}insert diagram of building block 3 here\textgreater{}

\paragraph{Building Block Name 3.1 (Black Box Description)}

\paragraph{Building Block Name 3.2 (Black Box Description)}

\paragraph{\ldots{}}

\paragraph{Building Block Name 3.n (Black Box Description)}

\paragraph{Description of Relationships}

\paragraph{Open Issues}

\subsection{Level 3}

\section{Runtime View}

\subsection{Runtime Scenario 1}

\subsection{Runtime Scenario 2}

\subsection{\ldots{}}

some more

\subsection{Runtime Scenario n}

\section{Deployment View}

\subsection{Infrastructure Level 1}

\subsubsection{Deployment Diagram Level 1}

\subsubsection{Processor 1}

\textless{}insert node template here\textgreater{}

\subsubsection{Processor 2}

\textless{}insert node template here\textgreater{}

\subsubsection{\ldots{}}

\subsubsection{Processor n}

\textless{}insert node template here\textgreater{}

\subsubsection{Channel 1}

\subsubsection{Channel 2}

\subsubsection{\ldots{}}

\subsubsection{Channel m}

\subsection{Infrastructure Level 2}

\section{Concepts}

\subsection{Domain Models}

This section is designed to give a short overview of the components of the app and how they interact.

The app is split into a number of background services and controllers. Each controller handles a certain aspect of the application such as TagCtrl handling tags and GraphControl handling the graph. Since the domain model is rather simple this section is rather 

\begin{figure}
\center
\begin{tikzpicture}
\umlclass{TagCtrl}{tags : list<SensorTag>}{connectToTag \\ scanForTags \\ onSensorData}
\umlclass[x=6]{SensorTag}{}{getSensorData}
\umlclass[x=-7]{DataStore}{}{storeData \\ retrieveData}
\umlaggreg[arg1=1, arg2=0..2]{TagCtrl}{SensorTag}
\umlassoc[arg2=storesData]{TagCtrl}{DataStore}
\end{tikzpicture}
\caption{This shows the relation between the TagCtrl and the SensorTags. The DataStore service is explained in more detail in below in the section about persistence}
\label{figTagCtrl}
\end{figure}

The TagCtrl controls the direct interaction with the sensor tag. It can scan for tags and connect to them. It is also the handler for all data retrieved from the tags. The relationship is shown in figure \ref{figTagCtrl}.

\begin{figure}
\center
\begin{tikzpicture}
\umlclass{GraphCtrl}{}{drawGraph \\ }
\umlclass[x=-7]{DataStore}{}{storeData \\ retrieveData}
\umlassoc[arg1=retrievesData]{GraphCtrl}{DataStore}
\end{tikzpicture}
\caption{This shows the relation between the GraphControl and the DataStore. The DataStore service is explained in more detail in below in the section about persistence}
\label{figGraphCtrl}
\end{figure}

The GraphCtrl retrieves data from the DataStore to draw informative graphs for the user.


\subsection{Recurring or Generic Structures and Patterns}

\subsubsection{Recurring or Generic Structure 1}

\textless{}insert diagram and descriptions here\textgreater{}

\subsubsection{Recurring or Generic Structure 2}

\textless{}insert diagram and descriptions here\textgreater{}

\subsection{Persistency}

There are two things that require persistency in the application: Settings and recorded data.

The settings are important to persist so that the user has a consistent experience and can adjust the application to his/her preferences. However the preferences are rather simple key-value pairs which don't need sophisticated storage.

Since the application is built on web technologies the HTML5 local data store is used for simple key value pair storage. This is a simple but easy to use method with little to no additional setup.

The recorded data however requires a different storage scheme as it can have multiple values associated with a single key and sometimes explicitly needs to define a type of the stored data. Also a much larger volume of data needs to be persisted. 

This is solved by using a file backed SQLite database which can be queried using SQL statements. This allows for very flexible data storage and is available on both iOS and Android platforms.

\subsection{User Interface}

The user interface is designed as multiple tabs with separation of concerns between the tabs. Each tab has an associated controller which only interacts with the other tabs via the background services. 

\begin{tabular}{|c|c|c|}
\hline
 Tab & Function\\
 \hline
 Tag & Search Tags, Establish Connection \\
 Graph & Visualize Information retrieved by Tags \\
 Settings & Allow for user adjustment of runtime parameters\\
 \hline
\end{tabular}

\subsection{Ergonomics}

\subsection{Flow of Control}

\subsection{Transaction Processing}

\subsection{Session Handling}

\subsection{Security}

\subsection{Safety}

\subsection{Communications and Integration}

\subsection{Distribution}

\subsection{Plausibility and Validity Checks}

\subsection{Exception/Error Handling}

Exceptions can occur during runtime either due to malformed javascript code as it is an interpreted language or due to underlying OS errors. In general these need to be handled differently.

The underlying OS errors should be handled by the respective plugins which interface directly with the OS. Handling those at the application level is not possible unless they are communicated as errors to the application by the plugin. An uncaught exception at the OS level terminates the program.

Javascript exceptions on the other hand can be caught using try catch functionality. However this is not done anywhere in the code.

Error conditions which can result from OS errors during plugin execution or malformed user input are checked for and reported to the user in the form of notifications. In the case of nonrecoverable errors the program is terminated after informing the user.

\subsection{System Management and Administration}

\subsection{Logging, Tracing}

There is a logging service which cumulates logs from multiple locations in the application. The application logs what it is currently doing allowing developers to trace back errors to recent events. These logs are accesible using the debug interface and are not meant for the end user. 

\subsection{Business Rules}

\subsection{Configurability}

The application has several parameters such as length of BLE scan, language etc. which are customizable by the user using the settings tab. 

Those settings are stored in the settings key-value store described above.

\subsection{Parallelization and Threading}

\subsection{Internationalization}

The application has full support for multiple languages as well as left to right languages. The language can be selected by the user in the settings tab.

\subsection{Migration}

\subsection{Testability}

There are both unit tests of each controller and background service as well as end to end tests. 

The unit tests can only test a small subset of the functionality while the end to end tests can test the interconnection between the various components.

\subsection{Scaling, Clustering}

\subsection{High Availability}

\subsection{Code Generation}

\subsection{Build-Management}

Builds can be fully automated using the provided build scripts which also automatically run all tests at build time. 

Current build automation uses Travis CI.

\section{Design Decisions}

For now see other docs.

\subsection{Decision Topic Template}

\subsection{Decision Topic 1}

\textbf{Decision.}

+

\subsection{Decision Topic 2}

+

\subsection{Decision Topic 3}

+

\subsection{\ldots{}}

+

\section{Quality Scenarios}

\subsection{Quality Tree}

\subsection{Evaluation Scenarios}

\section{Technical Risks}

\section{Glossary}

\begin{longtable}[c]{@{}llll@{}}
\toprule\addlinespace
\begin{minipage}[b]{0.22\columnwidth}\raggedright
Glossary
\end{minipage} & \begin{minipage}[b]{0.22\columnwidth}\raggedright
\end{minipage} & \begin{minipage}[b]{0.22\columnwidth}\raggedright
\end{minipage} & \begin{minipage}[b]{0.22\columnwidth}\raggedright
\end{minipage}
\\\addlinespace
\midrule\endhead
\begin{minipage}[t]{0.22\columnwidth}\raggedright
Term
\end{minipage} & \begin{minipage}[t]{0.22\columnwidth}\raggedright
Synonym
\end{minipage} & \begin{minipage}[t]{0.22\columnwidth}\raggedright
Description
\end{minipage} & \begin{minipage}[t]{0.22\columnwidth}\raggedright
\end{minipage}
\\\addlinespace
\begin{minipage}[t]{0.22\columnwidth}\raggedright
Term
\end{minipage} & \begin{minipage}[t]{0.22\columnwidth}\raggedright
Synonym
\end{minipage} & \begin{minipage}[t]{0.22\columnwidth}\raggedright
Description
\end{minipage} & \begin{minipage}[t]{0.22\columnwidth}\raggedright
\end{minipage}
\\\addlinespace
\bottomrule
\end{longtable}

\section{Literature and references}

\begin{description}
\item[Starke-2014]
Gernot Starke: Effektive Softwarearchitekturen - Ein praktischer
Leitfaden. Carl Hanser Verlag, 6, Auflage 2014.
\item[Starke-Hruschka-2011]
Gernot Starke und Peter Hruschka: Softwarearchitektur kompakt. Springer
Akademischer Verlag, 2. Auflage 2011.
\item[Zörner-2013]
Softwarearchitekturen dokumentieren und kommunizieren, Carl Hanser
Verlag, 2012
\end{description}

\section{Examples}

\begin{itemize}
\item
  \href{http://aim42.github.io/htmlSanityCheck/hsc_arc42.html}{HTML
  Sanity Checker}
\item
  \href{http://www.dokchess.de/dokchess/arc42/}{DocChess} (german)
\item
  \href{http://www.embarc.de/arc42-starschnitt-gradle/}{Gradle} (german)
\item
  \href{http://confluence.arc42.org/display/arc42beispielmamacrm}{MaMa
  CRM} (german)
\item
  \href{http://confluence.arc42.org/display/migrationEg/Financial+Data+Migration}{Financial
  Data Migration} (german)
\end{itemize}

\section{Acknowledgements and collaborations}

arc42 originally envisioned by \href{http://b-agile.de}{Dr. Peter
Hruschka} and \href{http://gernotstarke.de}{Dr. Gernot Starke}.

\begin{description}
\item[Sources]
We maintain arc42 in \emph{asciidoc} format at the moment, hosted in
\href{https://github.com/aim42/aim42}{GitHub under the
aim42-Organisation}.
\item[Issues]
We maintain a list of
\href{https://github.com/arc42/arc42-template/issues}{open topics and
bugs}.
\end{description}

We are looking forward to your corrections and clarifications! Please
fork the repository mentioned over this lines and send us a \emph{pull
request}!

\subsection{Collaborators}

We are very thankful and acknowledge the support and help provided by
all active and former collaborators, uncountable (anonymous) advisors,
bug finders and users of this method.

\subsubsection{Currently active}

\begin{itemize}
\item
  Gernot Starke
\item
  Stefan Zörner
\item
  Markus Schärtel
\item
  Ralf D. Müller
\item
  Peter Hruschka
\item
  Jürgen Krey
\end{itemize}

\subsubsection{Former collaborators}

(in alphabetical order)

\begin{itemize}
\item
  Anne Aloysius
\item
  Matthias Bohlen
\item
  Karl Eilebrecht
\item
  Manfred Ferken
\item
  Phillip Ghadir
\item
  Carsten Klein
\item
  Prof. Arne Koschel
\item
  Axel Scheithauer
\end{itemize}

\section{Approved Practitioner for arc42}

(TODO)

\end{document}
